% GNN Architecture Diagram
% Shows graph input -> GNN layer -> city embedding vector

\begin{tikzpicture}[scale=1.2]
    
    % Define coordinates
    \coordinate (graph-center) at (0,0);
    \coordinate (gnn-center) at (5,0);
    \coordinate (vector-pos) at (10,0);
    
    % Draw road network graph on the left
    \begin{scope}[shift={(graph-center)}]
        % Create a grid-like road network with intersections
        % Define nodes manually to avoid decimal naming issues
        \node[circle, fill=primaryblue, inner sep=1.5pt] (n11) at (-1.2, -1.2) {};
        \node[circle, fill=primaryblue, inner sep=1.5pt] (n12) at (-1.2, -0.4) {};
        \node[circle, fill=primaryblue, inner sep=1.5pt] (n13) at (-1.2, 0.4) {};
        \node[circle, fill=primaryblue, inner sep=1.5pt] (n14) at (-1.2, 1.2) {};
        
        \node[circle, fill=primaryblue, inner sep=1.5pt] (n21) at (-0.4, -1.2) {};
        \node[circle, fill=primaryblue, inner sep=1.5pt] (n22) at (-0.4, -0.4) {};
        \node[circle, fill=primaryblue, inner sep=1.5pt] (n23) at (-0.4, 0.4) {};
        \node[circle, fill=primaryblue, inner sep=1.5pt] (n24) at (-0.4, 1.2) {};
        
        \node[circle, fill=primaryblue, inner sep=1.5pt] (n31) at (0.4, -1.2) {};
        \node[circle, fill=primaryblue, inner sep=1.5pt] (n32) at (0.4, -0.4) {};
        \node[circle, fill=primaryblue, inner sep=1.5pt] (n33) at (0.4, 0.4) {};
        \node[circle, fill=primaryblue, inner sep=1.5pt] (n34) at (0.4, 1.2) {};
        
        \node[circle, fill=primaryblue, inner sep=1.5pt] (n41) at (1.2, -1.2) {};
        \node[circle, fill=primaryblue, inner sep=1.5pt] (n42) at (1.2, -0.4) {};
        \node[circle, fill=primaryblue, inner sep=1.5pt] (n43) at (1.2, 0.4) {};
        \node[circle, fill=primaryblue, inner sep=1.5pt] (n44) at (1.2, 1.2) {};
        
        % Horizontal road connections
        \draw[thick, primaryblue] (n11) -- (n21) -- (n31) -- (n41);
        \draw[thick, primaryblue] (n12) -- (n22) -- (n32) -- (n42);
        \draw[thick, primaryblue] (n13) -- (n23) -- (n33) -- (n43);
        \draw[thick, primaryblue] (n14) -- (n24) -- (n34) -- (n44);
        
        % Vertical road connections
        \draw[thick, primaryblue] (n11) -- (n12) -- (n13) -- (n14);
        \draw[thick, primaryblue] (n21) -- (n22) -- (n23) -- (n24);
        \draw[thick, primaryblue] (n31) -- (n32) -- (n33) -- (n34);
        \draw[thick, primaryblue] (n41) -- (n42) -- (n43) -- (n44);
        
        % Add some diagonal connections to make it more realistic
        \draw[thick, primaryblue] (n11) -- (n22);
        \draw[thick, primaryblue] (n33) -- (n44);
        \draw[thick, primaryblue] (n21) -- (n32);
        \draw[thick, primaryblue] (n12) -- (n23);
        \draw[thick, primaryblue] (n34) -- (n43);
        
        % Add some additional random connections to simulate non-grid roads
        \draw[thick, primaryblue] (n12) -- (n33);
        \draw[thick, primaryblue] (n21) -- (n42);
        \draw[thick, primaryblue] (n14) -- (n32);
        
        % Graph label
        \node[below] at (0, -1.5) {\small Road Network};
    \end{scope}
    
    % Arrow from graph to GNN
    \draw[arrow, very thick, primaryred] (1.5, 0) -- (3.5, 0);
    
        % GNN layer as stacked parallelogram blocks in perspective
    \begin{scope}[shift={(gnn-center)}]
        % Parameters for parallelograms
        \def\w{4 / 3}   % width
        \def\h{2}   % height
        \def\s{2 / 3}   % slant
        
        % GNN block - centered vertically
        \begin{scope}[shift={(-\w/2, -\h/2)}]
            \fill[primarygreen!50] (0,0) -- (\w,\s) -- (\w,{\h + \s}) -- (0,\h) -- cycle;
            \draw[primarygreen, thick] (0,0) -- (\w,\s) -- (\w,{\h + \s}) -- (0,\h) -- cycle;
            \node[font=\scriptsize] at (\w/2, {\h/2 + \s/2}) {GATv2};
        \end{scope}
        
        \node[below] at (0, -1.5) {\small GNN Layers};
    \end{scope}
    
    % Arrow from GNN to vector
    \draw[arrow, very thick, primaryred] (6.5, 0) -- (8.5, 0);
    
    % City embedding vector
    \begin{scope}[shift={(vector-pos)}]
        % Vector representation as vertical array
        \draw[thick, primarypurple] (-0.3, -1) rectangle (0.3, 1);
        
        % Vector elements (small rectangles)
        \foreach \y in {-0.8, -0.4, 0, 0.4, 0.8} {
            \fill[primarypurple!70] (-0.2, \y-0.1) rectangle (0.2, \y+0.1);
        }
        
        % Vector label
        \node[right, font=\small] at (0.4, 0) {$\mathbf{c}_{\mathrm{city}}$};
        \node[below] at (0, -1.3) {\small City Embedding};
    \end{scope}
    
\end{tikzpicture}
